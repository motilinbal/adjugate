\documentclass[12pt]{article}
\usepackage{amsmath}
\usepackage{amssymb}
\usepackage{amsthm}
\usepackage{geometry}
\geometry{a4paper, margin=1in}

\newtheorem{theorem}{Theorem}[section]
\newtheorem{problem}{Problem}[section]
\newtheorem{question}{Question}[problem]
\theoremstyle{definition}
\newtheorem{hint}{Hint}[question]

\newcommand{\adj}{\operatorname{adj}}
\newcommand{\detm}{\operatorname{det}}
\newcommand{\rank}{\operatorname{rank}}
\newcommand{\tr}{\operatorname{tr}}
\newcommand{\R}{\mathbb{R}}
\newcommand{\C}{\mathbb{C}}

\title{Classical Adjoint (Adjugate) Matrix: Problem Set}
\author{}
\date{\today}

\begin{document}

\maketitle

\section{Introduction: Defining the Classical Adjoint}
\subsection{Formal Definition and Notation}

\begin{problem}{Understanding Minors and Cofactors}
    Let $A = \begin{pmatrix} 1 & 0 & 2 \\ -1 & 3 & 1 \\ 0 & 1 & -2 \end{pmatrix}$.
    \begin{question}
        Calculate the minor $M_{23}$ and the cofactor $C_{23}$ of $A$[cite: 4, 5, 6].
    \end{question}
    
        \begin{hint}
            Recall that the minor $M_{ij}$ is the determinant of the submatrix formed by removing the $i$-th row and $j$-th column of $A$[cite: 5]. Which row and column should you remove for $M_{23}$?
        \end{hint}
        \begin{hint}
            After removing the 2nd row and 3rd column, what $2 \times 2$ matrix remains? Calculate its determinant. This is $M_{23}$[cite: 5].
        \end{hint}
        \begin{hint}
            The cofactor $C_{ij}$ is defined as $C_{ij} = (-1)^{i+j} M_{ij}$[cite: 6]. Use the value of $M_{23}$ you found and the appropriate sign based on $i=2, j=3$ to find $C_{23}$[cite: 6, 11].
        \end{hint}
    

    \begin{question}
        Explain why the entry in the $i$-th row and $j$-th column of $\adj(A)$ is $C_{ji}$, not $C_{ij}$[cite: 8, 9, 10].
    \end{question}
    
        \begin{hint}
            Start with the definition of the cofactor matrix $C$, whose $(i, j)$-th entry is $C_{ij}$[cite: 7].
        \end{hint}
        \begin{hint}
            How is the adjugate matrix $\adj(A)$ formally defined in terms of the cofactor matrix $C$? [cite: 8]
        \end{hint}
        \begin{hint}
            What operation transforms a matrix $M$ such that its $(i, j)$-th entry becomes the $(j, i)$-th entry of the original matrix? How does this apply to the relationship between $C$ and $\adj(A)$? [cite: 8, 9]
        \end{hint}
        \begin{hint}
            Consider the fundamental identity $A \cdot \adj(A) = \detm(A) I$[cite: 29]. Think about the dot product of the $i$-th row of $A$ with the $j$-th column of $\adj(A)$. What elements from the definition of $\adj(A)$ does this involve? [cite: 30, 31, 32] Does using $C_{ji}$ in the $(i, j)$ position of $\adj(A)$ lead to the correct identity?
        \end{hint}
    
\end{problem}

\section{Basic Computation of the Adjoint}
\subsection{Adjoint of a 2x2 Matrix}

\begin{problem}{Computing 2x2 Adjoints}
    \begin{question}
        Let $A = \begin{pmatrix} \cos \theta & -\sin \theta \\ \sin \theta & \cos \theta \end{pmatrix}$. Compute $\adj(A)$[cite: 12, 15, 16]. What does the result represent geometrically?
    \end{question}
    
        \begin{hint}
            Recall the shortcut for finding the adjugate of a $2 \times 2$ matrix $A = \begin{pmatrix} a & b \\ c & d \end{pmatrix}$[cite: 16]. What happens to the diagonal elements? What happens to the off-diagonal elements?
        \end{hint}
        \begin{hint}
            Apply the shortcut to the given matrix $A$ with $a = \cos \theta$, $b = -\sin \theta$, $c = \sin \theta$, $d = \cos \theta$[cite: 15, 16].
        \end{hint}
        \begin{hint}
            Compare the resulting matrix $\adj(A)$ with the original matrix $A$ and its transpose $A^T$. What transformation does $A$ represent? What transformation does $\adj(A)$ represent? (Think about rotations).
        \end{hint}
         \begin{hint}
             Consider the determinant of $A$. How does $\adj(A)$ relate to $A^{-1}$ when $A$ represents a rotation? [cite: 38, 130]
         \end{hint}
    

    \begin{question}
        Verify the formula $\adj(A) = \begin{pmatrix} d & -b \\ -c & a \end{pmatrix}$ for $A = \begin{pmatrix} a & b \\ c & d \end{pmatrix}$ by explicitly calculating the cofactor matrix $C$ and transposing it[cite: 13, 14, 15].
    \end{question}
        
        \begin{hint}
            Calculate the four minors $M_{11}, M_{12}, M_{21}, M_{22}$[cite: 13]. Remember $M_{ij}$ is the determinant of the $1 \times 1$ matrix obtained by removing row $i$ and column $j$. The determinant of $(x)$ is just $x$.
        \end{hint}
        \begin{hint}
            Calculate the four cofactors $C_{11}, C_{12}, C_{21}, C_{22}$ using $C_{ij} = (-1)^{i+j} M_{ij}$ and form the cofactor matrix $C$[cite: 6, 14]. Pay attention to the signs!
        \end{hint}
        \begin{hint}
            Find the transpose of the cofactor matrix $C$. Does it match the shortcut formula for $\adj(A)$? [cite: 8, 15]
        \end{hint}
    
\end{problem}

\subsection{Adjoint of a 3x3 Matrix}

\begin{problem}{Computing a 3x3 Adjoint}
    \begin{question}
        Calculate $\adj(A)$ for $A = \begin{pmatrix} 1 & 0 & 1 \\ 0 & 2 & 0 \\ -1 & 0 & 1 \end{pmatrix}$[cite: 20, 21, 22, 23, 24].
    \end{question}
    
        \begin{hint}
            First, find the $3 \times 3$ matrix of minors. For each element $A_{ij}$, calculate $M_{ij}$ by finding the determinant of the $2 \times 2$ submatrix obtained by removing row $i$ and column $j$[cite: 20, 21]. For example, $M_{11} = \detm \begin{pmatrix} 2 & 0 \\ 0 & 1 \end{pmatrix}$. Calculate all nine minors[cite: 22].
        \end{hint}
        \begin{hint}
            Next, find the cofactor matrix $C$. Apply the checkerboard pattern of signs $+ - +, - + -, + - +$ to the matrix of minors, or equivalently, compute $C_{ij} = (-1)^{i+j} M_{ij}$ for each entry[cite: 22, 23].
        \end{hint}
        \begin{hint}
            Finally, the adjugate matrix $\adj(A)$ is the transpose of the cofactor matrix $C$[cite: 24]. Swap the rows and columns of $C$.
        \end{hint}
        \begin{hint}
            Double-check your calculation by computing $A \cdot \adj(A)$. Does it equal $\detm(A) I_3$? [cite: 29] First compute $\detm(A)$.
        \end{hint}
    
\end{problem}

\begin{problem}{Adjoint of a Singular Matrix}
    \begin{question}
        Let $A = \begin{pmatrix} 1 & 2 & 3 \\ 4 & 5 & 6 \\ 7 & 8 & 9 \end{pmatrix}$[cite: 25]. Calculate $\adj(A)$. What do you notice about the columns (or rows) of $\adj(A)$? Relate this to the determinant and rank of $A$.
    \end{question}
    
        \begin{hint}
            Follow the three steps: Calculate the matrix of minors, apply signs to get the cofactor matrix, and transpose to get the adjugate[cite: 20, 22, 24].
        \end{hint}
        \begin{hint}
            Notice that calculating the minors might yield some zeros or simple relationships. For instance, $M_{11} = 5 \times 9 - 6 \times 8 = 45 - 48 = -3$. $M_{12} = 4 \times 9 - 6 \times 7 = 36 - 42 = -6$. $M_{13} = 4 \times 8 - 5 \times 7 = 32 - 35 = -3$. Continue for all nine minors.
        \end{hint}
        \begin{hint}
            After finding $\adj(A)$, observe its columns. Are they related? Are they scalar multiples of each other? [cite: 85]
        \end{hint}
        \begin{hint}
            Calculate $\detm(A)$. You can use cofactor expansion or notice that the columns (or rows) are linearly dependent (e.g., Col2 - Col1 = Col3 - Col2). What does $\detm(A)=0$ imply about $A \cdot \adj(A)$? [cite: 29, 39]
        \end{hint}
        \begin{hint}
             What is the rank of $A$? (It's not 3 since $\detm(A)=0$. Is it 2 or 1?)[cite: 81]. How does the rank of $A$ relate to the rank of $\adj(A)$? Does your calculated $\adj(A)$ have the expected rank? [cite: 80, 81, 82]
        \end{hint}
    
\end{problem}

\section{Fundamental Properties and Theorems}
\subsection{The Fundamental Identity}

\begin{problem}{Applying the Core Identity}
    \begin{question}
        Let $A$ be a $3 \times 3$ matrix with $\detm(A) = 5$. Without calculating $A$ or $\adj(A)$ explicitly, find the matrix product $A \cdot \adj(A)$.
    \end{question}
    
        \begin{hint}
            Recall the fundamental identity relating $A$, $\adj(A)$, and $\detm(A)$[cite: 29].
        \end{hint}
        \begin{hint}
            The identity is $A \cdot \adj(A) = \detm(A) I_n$[cite: 29]. What is $n$ in this case? What is $I_n$?
        \end{hint}
        \begin{hint}
            Substitute the given value of $\detm(A)$ and the appropriate identity matrix $I_n$ into the formula.
        \end{hint}
    

    \begin{question}
        Suppose $A$ is an $n \times n$ matrix such that $A \cdot \adj(A) = \begin{pmatrix} -2 & 0 \\ 0 & -2 \end{pmatrix}$. What is $\detm(A)$? Is $A$ invertible? [cite: 29, 38]
    \end{question}
    
        \begin{hint}
             Compare the given matrix product with the general form $A \cdot \adj(A) = \detm(A) I_n$[cite: 29]. What must $\detm(A)$ be? What is $n$?
        \end{hint}
        \begin{hint}
             A matrix is invertible if and only if its determinant is non-zero[cite: 38]. Based on the value you found for $\detm(A)$, can you conclude whether $A$ is invertible?
        \end{hint}
    
\end{problem}

\subsection{Invertibility and the Adjoint}

\begin{problem}{Adjoint and Inverse}
    \begin{question}
        Given $A = \begin{pmatrix} 1 & 2 \\ 3 & 4 \end{pmatrix}$. Find $A^{-1}$ using the formula involving the adjugate[cite: 38].
    \end{question}
    
        \begin{hint}
            The formula is $A^{-1} = \frac{1}{\detm(A)} \adj(A)$, provided $\detm(A) \neq 0$[cite: 38]. First, calculate $\detm(A)$. Is it non-zero?
        \end{hint}
        \begin{hint}
            Next, calculate $\adj(A)$ using the shortcut for $2 \times 2$ matrices[cite: 16].
        \end{hint}
        \begin{hint}
            Substitute $\detm(A)$ and $\adj(A)$ into the inverse formula.
        \end{hint}
    

    \begin{question}
        Let $A$ be an invertible $n \times n$ matrix. Express $\adj(A)$ in terms of $A^{-1}$ and $\detm(A)$[cite: 38, 41]. Use this to argue why $\adj(A)$ must also be invertible.
    \end{question}
    
        \begin{hint}
            Start with the formula for the inverse: $A^{-1} = \frac{1}{\detm(A)} \adj(A)$[cite: 38].
        \end{hint}
        \begin{hint}
            Rearrange the formula algebraically to solve for $\adj(A)$. Remember $\detm(A)$ is just a non-zero scalar since $A$ is invertible.
        \end{hint}
        \begin{hint}
            You should arrive at $\adj(A) = \detm(A) A^{-1}$[cite: 41]. Since $A$ is invertible, $A^{-1}$ exists. Is the product of a non-zero scalar and an invertible matrix also invertible? Why?
        \end{hint}
        \begin{hint}
             Alternatively, consider the determinant of $\adj(A)$. We know $\detm(\adj(A)) = (\detm A)^{n-1}$[cite: 57]. If $A$ is invertible, what does this tell you about $\detm(\adj(A))$ (assuming $n>1$)? [cite: 40]
        \end{hint}
    

    \begin{question}
        If $A$ is a singular $3 \times 3$ matrix, can $\adj(A)$ be invertible? Justify your answer using the relationship between the rank of $A$ and the rank of $\adj(A)$[cite: 39, 40, 42, 80, 81, 82]. Consider the special case $n=1$ separately[cite: 44].
    \end{question}
    
        \begin{hint}
            What does it mean for $A$ to be singular? [cite: 39] What are the possible ranks for a singular $3 \times 3$ matrix $A$?
        \end{hint}
        \begin{hint}
            Recall the rules relating $\rank(A)$ and $\rank(\adj(A))$ for an $n \times n$ matrix[cite: 80]:
            If $\rank(A) = n$, then $\rank(\adj(A)) = n$.
            If $\rank(A) = n-1$, then $\rank(\adj(A)) = 1$.
            If $\rank(A) \leq n-2$, then $\rank(\adj(A)) = 0$.
        \end{hint}
        \begin{hint}
            Apply these rules for $n=3$ and the possible ranks of the singular matrix $A$. What are the possible ranks for $\adj(A)$?
        \end{hint}
        \begin{hint}
            For a matrix to be invertible, what must its rank be? Can $\adj(A)$ have this rank if $A$ is singular and $n=3$? [cite: 40]
        \end{hint}
         \begin{hint}
             What happens in the $n=1$ case? If $A=(0)$, what is $\adj(A)$? Is it invertible? [cite: 44, 141]
         \end{hint}
    
\end{problem}

\section{Matrix Identities Involving the Adjoint}

\begin{problem}{Exploring Adjoint Identities}
    \begin{question}
        Prove that $\adj(A^T) = (\adj(A))^T$ for any $n \times n$ matrix $A$[cite: 45].
    \end{question}
    
        \begin{hint}
            Let $B = A^T$. We want to show $\adj(B) = (\adj(A))^T$. Let's look at the $(i, j)$-th entry of each side. The $(i, j)$-th entry of $(\adj(A))^T$ is the $(j, i)$-th entry of $\adj(A)$. What is this entry by definition? [cite: 9]
        \end{hint}
        \begin{hint}
            The $(j, i)$-th entry of $\adj(A)$ is $C_{ij}(A) = (-1)^{i+j} M_{ij}(A)$, where the cofactor and minor are calculated from $A$[cite: 9].
        \end{hint}
        \begin{hint}
            Now consider the $(i, j)$-th entry of $\adj(B) = \adj(A^T)$. By definition, this is $C_{ji}(B) = (-1)^{j+i} M_{ji}(B)$, where the cofactor and minor are calculated from $B=A^T$[cite: 9].
        \end{hint}
        \begin{hint}
            How does the minor $M_{ji}(B)$ relate to a minor of $A$? Recall that $B=A^T$. $M_{ji}(B)$ is the determinant of the matrix obtained by removing row $j$ and column $i$ from $A^T$[cite: 46]. How does this submatrix relate to the submatrix used to calculate $M_{ij}(A)$ (obtained by removing row $i$ and column $j$ from $A$)? [cite: 46]
        \end{hint}
        \begin{hint}
            Remember that $\detm(M) = \detm(M^T)$[cite: 46]. Use this to show that $M_{ji}(B) = M_{ij}(A)$.
        \end{hint}
        \begin{hint}
            Substitute $M_{ji}(B) = M_{ij}(A)$ into the expression for the $(i, j)$-th entry of $\adj(B)$ and compare it to the expression for the $(i, j)$-th entry of $(\adj(A))^T$. Are they equal? [cite: 47]
        \end{hint}
    

    \begin{question}
        Let $A$ and $B$ be $3 \times 3$ matrices with $\detm(A) = 2$ and $\detm(B) = -3$. Calculate $\detm(\adj(AB))$.
    \end{question}
    
        \begin{hint}
            We need $\detm(\adj(M))$ where $M=AB$. Recall the identity relating the determinant of the adjugate to the determinant of the original matrix: $\detm(\adj(M)) = (\detm M)^{n-1}$[cite: 57]. What is $n$ here?
        \end{hint}
        \begin{hint}
            First, find $\detm(M) = \detm(AB)$. How does the determinant of a product relate to the determinants of the factors? [cite: 51]
        \end{hint}
        \begin{hint}
            Calculate $\detm(AB)$ using the given determinants of $A$ and $B$.
        \end{hint}
        \begin{hint}
            Now substitute this value for $\detm(M)$ and $n=3$ into the formula $\detm(\adj(M)) = (\detm M)^{n-1}$.
        \end{hint}
        \begin{hint}
            Alternatively, use $\adj(AB) = \adj(B)\adj(A)$[cite: 49]. Then $\detm(\adj(AB)) = \detm(\adj(B)\adj(A))$. How does the determinant behave with products?
        \end{hint}
        \begin{hint}
            $\detm(\adj(B)\adj(A)) = \detm(\adj(B)) \detm(\adj(A))$. Now use $\detm(\adj(X)) = (\detm X)^{n-1}$ for $X=A$ and $X=B$[cite: 57]. Does this give the same result?
        \end{hint}
    

    \begin{question}
        Let $A$ be a $4 \times 4$ matrix and $c=2$. If $\adj(A) = M$, what is $\adj(cA)$ in terms of $M$ and $c$? [cite: 54]
    \end{question}
    
        \begin{hint}
            Recall the identity for the adjugate of a scalar multiple: $\adj(cA) = c^{n-1} \adj(A)$[cite: 54].
        \end{hint}
        \begin{hint}
            Identify the values of $c$ and $n$ in this problem.
        \end{hint}
        \begin{hint}
            Substitute these values and the fact that $\adj(A)=M$ into the identity.
        \end{hint}
    

    \begin{question}
        If $A$ is a $3 \times 3$ matrix with $\detm(A) = 4$, find $\detm(\adj(\adj(A)))$.
    \end{question}
    
        \begin{hint}
            Let $B = \adj(A)$. We want to find $\detm(\adj(B))$. Use the identity $\detm(\adj(B)) = (\detm B)^{n-1}$[cite: 57]. What is $n$?
        \end{hint}
        \begin{hint}
            We need $\detm(B) = \detm(\adj(A))$. Use the same identity again: $\detm(\adj(A)) = (\detm A)^{n-1}$[cite: 57]. Calculate this value using the given $\detm(A)$ and $n$.
        \end{hint}
        \begin{hint}
            Substitute the value of $\detm(B)$ you just found back into the expression from the first hint: $\detm(\adj(B)) = (\detm B)^{n-1}$.
        \end{hint}
        \begin{hint}
            Alternatively, use the identity $\adj(\adj(A)) = (\detm A)^{n-2} A$ for $n \ge 2$[cite: 67].
        \end{hint}
        \begin{hint}
             Take the determinant of both sides of $\adj(\adj(A)) = (\detm A)^{n-2} A$. Remember that $\detm(kA) = k^n \detm(A)$ for a scalar $k$. Let $k = (\detm A)^{n-2}$.
        \end{hint}
         \begin{hint}
             So, $\detm(\adj(\adj(A))) = \detm( (\detm A)^{n-2} A ) = ((\detm A)^{n-2})^n \detm(A)$. Simplify this expression using $n=3$ and $\detm(A)=4$. Does it match the previous result?
         \end{hint}
    

    \begin{question}
        Let $A$ be an invertible $n \times n$ matrix. Show that $(\adj(A))^{-1} = \adj(A^{-1})$.
    \end{question}
    
        \begin{hint}
            We want to show that $\adj(A) \cdot \adj(A^{-1}) = I$. Recall the identity $\adj(XY) = \adj(Y)\adj(X)$[cite: 49]. Apply this with $X=A$ and $Y=A^{-1}$.
        \end{hint}
        \begin{hint}
            What is $XY = A A^{-1}$? What is $\adj(A A^{-1})$? [cite: 141]
        \end{hint}
        \begin{hint}
            So we have $\adj(A^{-1})\adj(A) = \adj(I) = I$[cite: 49, 141]. Does this directly show $(\adj(A))^{-1} = \adj(A^{-1})$? Yes, by the definition of an inverse.
        \end{hint}
        \begin{hint}
            Alternatively, start with $A^{-1} = \frac{1}{\detm(A)} \adj(A)$[cite: 38]. And $\adj(A^{-1}) = \detm(A^{-1}) (A^{-1})^{-1}$[cite: 41].
        \end{hint}
        \begin{hint}
            We know $\detm(A^{-1}) = 1/\detm(A)$ and $(A^{-1})^{-1} = A$. Substitute these into the second equation: $\adj(A^{-1}) = \frac{1}{\detm(A)} A$.
        \end{hint}
        \begin{hint}
            Now, let's find $(\adj(A))^{-1}$. We know $\adj(A) = \detm(A) A^{-1}$[cite: 41]. So $(\adj(A))^{-1} = (\detm(A) A^{-1})^{-1}$. How does the inverse distribute over a scalar multiple and the matrix?
        \end{hint}
        \begin{hint}
            $(\detm(A) A^{-1})^{-1} = (\detm A)^{-1} (A^{-1})^{-1} = \frac{1}{\detm A} A$. Compare this with the expression for $\adj(A^{-1})$. Are they equal?
        \end{hint}
         \begin{hint}
             Another way: Use the identity $\adj(A^k) = (\adj(A))^k$ [cite: 76] with $k=-1$. Does this identity hold for negative integers if $A$ is invertible? Assume it does for a moment.
         \end{hint}
    
\end{problem}

\section{Relationship with Rank}

\begin{problem}{Rank Relationships}
    \begin{question}
        Let $A$ be a $5 \times 5$ matrix. Determine $\rank(\adj(A))$ in each of the following cases:
        (a) $\rank(A) = 5$
        (b) $\rank(A) = 4$
        (c) $\rank(A) = 3$
        (d) $\rank(A) = 0$ [cite: 80, 81, 82]
    \end{question}
    
        \begin{hint}
            Recall the rules governing the relationship between $\rank(A)$ and $\rank(\adj(A))$ for an $n \times n$ matrix[cite: 80]. What is $n$ here?
        \end{hint}
        \begin{hint}
            Case (a): If $\rank(A) = n$, what is $\rank(\adj(A))$? [cite: 80]
        \end{hint}
        \begin{hint}
            Case (b): If $\rank(A) = n-1$, what is $\rank(\adj(A))$? [cite: 81]
        \end{hint}
        \begin{hint}
            Case (c): If $\rank(A) \le n-2$, what is $\rank(\adj(A))$? Does $\rank(A)=3$ satisfy this condition for $n=5$? [cite: 82]
        \end{hint}
        \begin{hint}
            Case (d): If $A$ is the zero matrix (rank 0), what is $\adj(A)$ (assuming $n>1$)? What is its rank? [cite: 82, 140] Does this fit the general rule for $\rank(A) \le n-2$?
        \end{hint}
    

    \begin{question}
        Let $A$ be a $3 \times 3$ matrix with $\rank(A)=2$. Let $\mathbf{v}$ be a non-zero vector such that $A\mathbf{v} = \mathbf{0}$. Explain why every column of $\adj(A)$ must be a scalar multiple of $\mathbf{v}$[cite: 83, 84, 85, 86].
    \end{question}
    
        \begin{hint}
            If $\rank(A)=2$ for a $3 \times 3$ matrix, what is $\detm(A)$? [cite: 81, 83]
        \end{hint}
        \begin{hint}
            What does the fundamental identity $A \cdot \adj(A) = \detm(A) I$ become in this case? [cite: 39, 83]
        \end{hint}
        \begin{hint}
            Let $\mathbf{c}_j$ be the $j$-th column of $\adj(A)$. Consider the product $A \cdot \adj(A)$. The $j$-th column of this product is $A \mathbf{c}_j$. What does the result from the previous hint tell you about $A \mathbf{c}_j$? [cite: 84]
        \end{hint}
        \begin{hint}
            If $A \mathbf{c}_j = \mathbf{0}$, what space does the vector $\mathbf{c}_j$ belong to? [cite: 84]
        \end{hint}
        \begin{hint}
            What is the dimension of the nullspace (kernel) of $A$ if $A$ is $3 \times 3$ and $\rank(A)=2$? (Rank-Nullity Theorem)[cite: 85].
        \end{hint}
        \begin{hint}
            We are given that $A\mathbf{v} = \mathbf{0}$ with $\mathbf{v} \neq \mathbf{0}$. What does this mean about $\mathbf{v}$ in relation to the nullspace? Since the nullspace has dimension 1, how must any other vector in the nullspace (like the columns $\mathbf{c}_j$) relate to $\mathbf{v}$? [cite: 85, 86]
        \end{hint}
         \begin{hint}
             Also, what is the rank of $\adj(A)$ when $\rank(A)=n-1=2$?[cite: 81]. How does the rank relate to the columns spanning only a 1D space (the nullspace)? [cite: 86]
         \end{hint}
    
\end{problem}

\section{Applications}

\begin{problem}{Inverse and System Solving}
    \begin{question}
        Find the inverse of $A = \begin{pmatrix} 1 & 0 & 1 \\ 0 & 2 & 0 \\ -1 & 0 & 1 \end{pmatrix}$ using the adjugate method[cite: 87, 88]. (You may have computed $\adj(A)$ in Problem 2.2.1).
    \end{question}
    
        \begin{hint}
            First, compute $\detm(A)$. A cofactor expansion along the second row might be efficient here.
        \end{hint}
        \begin{hint}
            Retrieve or re-compute $\adj(A)$ using the standard procedure (minors, cofactors, transpose)[cite: 20, 24].
        \end{hint}
        \begin{hint}
            Apply the formula $A^{-1} = \frac{1}{\detm(A)} \adj(A)$[cite: 38, 87].
        \end{hint}
        \begin{hint}
            Verify your result by computing $A A^{-1}$. Should it equal $I_3$?
        \end{hint}
    

    \begin{question}
        Consider the system $A\mathbf{x} = \mathbf{b}$ where $A$ is an $n \times n$ singular matrix ($\detm(A)=0$). Show that if a solution $\mathbf{x}$ exists, it must satisfy $\adj(A) \mathbf{b} = \mathbf{0}$[cite: 97, 98, 99, 100]. Is the converse true (if $\adj(A) \mathbf{b} = \mathbf{0}$, does a solution necessarily exist)?
    \end{question}
    
        \begin{hint}
            Start with the assumption that a solution $\mathbf{x}$ exists, so $A\mathbf{x} = \mathbf{b}$ holds[cite: 97].
        \end{hint}
        \begin{hint}
            Multiply both sides of the equation $A\mathbf{x} = \mathbf{b}$ on the left by $\adj(A)$[cite: 98].
        \end{hint}
        \begin{hint}
            Use the fundamental identity $\adj(A) A = \detm(A) I_n$[cite: 29]. What does this become since $A$ is singular? [cite: 39, 99]
        \end{hint}
        \begin{hint}
            Substitute the result from the previous hint into the equation from Hint 2. What condition must $\adj(A)\mathbf{b}$ satisfy? [cite: 100]
        \end{hint}
        \begin{hint}
            For the converse: $\adj(A)\mathbf{b} = \mathbf{0}$ is a necessary condition for consistency[cite: 100]. Does it guarantee a solution exists? Think about the relationship between $\mathbf{b}$ and the column space of $A$. The condition $\adj(A)\mathbf{b} = \mathbf{0}$ implies that $\mathbf{b}$ is orthogonal to the rows of $\adj(A)$. How do the rows of $\adj(A)$ relate to the nullspace of $A^T$? What is the relationship between the nullspace of $A^T$ and the column space of $A$ (Fundamental Theorem of Linear Algebra)? [cite: 102]
        \end{hint}
         \begin{hint}
             Consider $A = \begin{pmatrix} 1 & 1 \\ 1 & 1 \end{pmatrix}$ and $\mathbf{b} = \begin{pmatrix} 1 \\ 0 \end{pmatrix}$. Calculate $\adj(A)$ and $\adj(A)\mathbf{b}$. Does a solution to $A\mathbf{x} = \mathbf{b}$ exist? (The column space of A is spanned by $\begin{pmatrix} 1 \\ 1 \end{pmatrix}$).
         \end{hint}
    
\end{problem}

\section{Properties Under Transformations and Specific Matrix Types}

\begin{problem}{Adjoints of Special Matrices}
    \begin{question}
        Let $A$ be a $3 \times 3$ orthogonal matrix with $\detm(A)=-1$. Find a simple expression for $\adj(A)$ in terms of $A$ or $A^T$[cite: 130, 131].
    \end{question}
    
        \begin{hint}
            What does it mean for $A$ to be orthogonal? What is the relationship between $A^{-1}$ and $A^T$? [cite: 130]
        \end{hint}
        \begin{hint}
            Recall the formula relating the adjugate and the inverse for an invertible matrix: $\adj(A) = \detm(A) A^{-1}$[cite: 128].
        \end{hint}
        \begin{hint}
            Substitute the relationship between $A^{-1}$ and $A^T$ (from Hint 1) and the given value of $\detm(A)$ into the formula from Hint 2[cite: 131].
        \end{hint}
    

    \begin{question}
        Prove that if $A$ is symmetric ($A^T=A$), then $\adj(A)$ is also symmetric[cite: 132].
    \end{question}
    
        \begin{hint}
            We want to show that $(\adj(A))^T = \adj(A)$.
        \end{hint}
        \begin{hint}
            Recall the identity relating the adjugate of a transpose and the transpose of the adjugate: $(\adj(A))^T = \adj(A^T)$[cite: 45, 132].
        \end{hint}
        \begin{hint}
            Since $A$ is symmetric, what is $A^T$ equal to? Substitute this into the right side of the identity from Hint 2. Does this prove the claim? [cite: 132]
        \end{hint}
    

    \begin{question}
        Let $A$ be an $n \times n$ skew-symmetric matrix ($A^T = -A$). Determine whether $\adj(A)$ is symmetric or skew-symmetric, considering whether $n$ is odd or even[cite: 133, 134, 135, 136, 137].
    \end{question}
    
        \begin{hint}
            We need to examine $(\adj(A))^T$. Start with the identity $(\adj(A))^T = \adj(A^T)$[cite: 45, 133].
        \end{hint}
        \begin{hint}
            Since $A$ is skew-symmetric, substitute $A^T = -A$ into the identity: $(\adj(A))^T = \adj(-A)$[cite: 133].
        \end{hint}
        \begin{hint}
            Recall the identity for the adjugate of a scalar multiple: $\adj(cA) = c^{n-1} \adj(A)$[cite: 54]. Apply this with $c=-1$: $\adj(-A) = (-1)^{n-1} \adj(A)$[cite: 133].
        \end{hint}
        \begin{hint}
            Combine the results: $(\adj(A))^T = (-1)^{n-1} \adj(A)$[cite: 133].
        \end{hint}
        \begin{hint}
            Case 1: $n$ is odd. What is $n-1$? What is $(-1)^{n-1}$? What does the equation from Hint 4 become? Does this mean $\adj(A)$ is symmetric or skew-symmetric? [cite: 135, 137]
        \end{hint}
        \begin{hint}
            Case 2: $n$ is even. What is $n-1$? What is $(-1)^{n-1}$? What does the equation from Hint 4 become? Does this mean $\adj(A)$ is symmetric or skew-symmetric? [cite: 136, 137]
        \end{hint}
    

    \begin{question}
        Let $D = \operatorname{diag}(d_1, d_2, \dots, d_n)$. Describe the matrix $\adj(D)$[cite: 138, 139, 140]. What is $\adj(D)$ if one of the $d_i$ is zero? What if at least two $d_i$'s are zero (assume $n > 1$)?
    \end{question}
    
        \begin{hint}
            Consider the $(i, j)$-th entry of $\adj(D)$, which is $C_{ji}(D) = (-1)^{j+i} M_{ji}(D)$[cite: 9].
        \end{hint}
        \begin{hint}
            What is the minor $M_{ji}(D)$ if $j \neq i$? The submatrix used to calculate $M_{ji}(D)$ is obtained by removing row $j$ and column $i$ from the diagonal matrix $D$. Does this submatrix still have zeros everywhere off the main diagonal? Does it have a zero on its main diagonal? What is its determinant? [cite: 138, 305]
        \end{hint}
         \begin{hint}
             This implies that $C_{ji}(D) = 0$ if $j \neq i$. So, $\adj(D)$ must be a diagonal matrix[cite: 138].
         \end{hint}
        \begin{hint}
            Now consider the diagonal entries $(i, i)$ of $\adj(D)$. These are $C_{ii}(D) = (-1)^{i+i} M_{ii}(D) = M_{ii}(D)$. What is the minor $M_{ii}(D)$? It's the determinant of the diagonal matrix formed by removing row $i$ and column $i$ from $D$[cite: 139].
        \end{hint}
        \begin{hint}
            The matrix for $M_{ii}(D)$ is $\operatorname{diag}(d_1, \dots, d_{i-1}, d_{i+1}, \dots, d_n)$. What is its determinant? [cite: 139] This is the $i$-th diagonal entry of $\adj(D)$[cite: 140].
        \end{hint}
        \begin{hint}
            If exactly one $d_k=0$. Consider the $k$-th diagonal entry of $\adj(D)$. It's the product $\prod_{j \neq k} d_j$. Is this zero or non-zero? Now consider any other diagonal entry $i \neq k$. Its value is $\prod_{j \neq i} d_j$. Does this product include $d_k$? What is its value? What does $\adj(D)$ look like?
        \end{hint}
         \begin{hint}
             If at least two entries, say $d_k=0$ and $d_l=0$ ($k \neq l$), consider any diagonal entry $\prod_{j \neq i} d_j$ of $\adj(D)$. Can this product ever be non-zero? (Think about whether $d_k$ or $d_l$ must be included in the product). What is $\adj(D)$ in this case? [cite: 82]
         \end{hint}
    
\end{problem}

\section{Conceptual, Geometric, and Advanced Topics}

\begin{problem}{Properties and Interpretations}
    \begin{question}
        Explain why the mapping $F: \R^{n \times n} \to \R^{n \times n}$ defined by $F(A) = \adj(A)$ is a continuous function[cite: 142, 143, 144, 145].
    \end{question}
    
        \begin{hint}
            What are the entries of the matrix $\adj(A)$? They are cofactors $C_{ji}(A)$[cite: 9].
        \end{hint}
        \begin{hint}
            How is a cofactor $C_{ji}(A)$ defined? It involves a minor $M_{ji}(A)$ and a sign[cite: 6].
        \end{hint}
        \begin{hint}
            How is a minor $M_{ji}(A)$ defined? It's the determinant of a submatrix of $A$[cite: 5].
        \end{hint}
        \begin{hint}
            Is the determinant function a polynomial function of the entries of a matrix? [cite: 144]
        \end{hint}
        \begin{hint}
            Therefore, are the minors polynomials in the entries of $A$? Are the cofactors? Are the entries of $\adj(A)$? [cite: 144]
        \end{hint}
        \begin{hint}
            Are polynomial functions continuous? If each entry of $F(A)$ is a continuous function of the entries of $A$, what does that imply about the continuity of the matrix function $F(A)$? [cite: 145]
        \end{hint}
    

    \begin{question}
        Let $A = \begin{pmatrix} \mathbf{a}_1 & \mathbf{a}_2 & \mathbf{a}_3 \end{pmatrix}$ be a $3 \times 3$ matrix with columns $\mathbf{a}_1, \mathbf{a}_2, \mathbf{a}_3 \in \R^3$. Verify the formula stating that the *rows* of $\adj(A)$ are given by $(\mathbf{a}_2 \times \mathbf{a}_3)^T$, $(\mathbf{a}_3 \times \mathbf{a}_1)^T$, and $(\mathbf{a}_1 \times \mathbf{a}_2)^T$[cite: 174, 175]. (Note: The source document discussion is slightly confusing, this formulation aligns with $A \adj(A) = \det(A)I$).
    \end{question}
    
        \begin{hint}
            Let $A = \begin{pmatrix} a_{11} & a_{12} & a_{13} \\ a_{21} & a_{22} & a_{23} \\ a_{31} & a_{32} & a_{33} \end{pmatrix}$. The first row of $\adj(A)$ consists of the cofactors $C_{11}, C_{21}, C_{31}$[cite: 9, 167].
        \end{hint}
        \begin{hint}
            Calculate the cofactor $C_{11} = (-1)^{1+1} M_{11}$. Write out the $2 \times 2$ determinant for $M_{11}$.
        \end{hint}
        \begin{hint}
            Now calculate the cross product $\mathbf{a}_2 \times \mathbf{a}_3$. Write out the components of this vector using the determinant formula for the cross product or component-wise definition.
        \end{hint}
        \begin{hint}
            Compare the first component of $\mathbf{a}_2 \times \mathbf{a}_3$ with the cofactor $C_{11}$. Are they equal?
        \end{hint}
        \begin{hint}
            Now calculate the cofactor $C_{21} = (-1)^{2+1} M_{21}$. Write out the $2 \times 2$ determinant for $M_{21}$. Compare it with the second component of $\mathbf{a}_2 \times \mathbf{a}_3$. Are they equal? (Pay attention to the sign from $(-1)^{2+1}$).
        \end{hint}
         \begin{hint}
             Calculate $C_{31} = (-1)^{3+1} M_{31}$. Compare it with the third component of $\mathbf{a}_2 \times \mathbf{a}_3$. Are they equal?
         \end{hint}
         \begin{hint}
             Since the components of $(\mathbf{a}_2 \times \mathbf{a}_3)^T$ match the first row of $\adj(A)$ ($C_{11}, C_{21}, C_{31}$), the first part of the formula is verified. Repeat the process for the second and third rows of $\adj(A)$ using $\mathbf{a}_3 \times \mathbf{a}_1$ and $\mathbf{a}_1 \times \mathbf{a}_2$ respectively.
         \end{hint}
    

    \begin{question}
       Let $\lambda$ be an eigenvalue of an invertible $n \times n$ matrix $A$. Show that $\detm(A)/\lambda$ is an eigenvalue of $\adj(A)$[cite: 111, 112, 113]. What happens if $A$ is singular?
    \end{question}
     
        \begin{hint}
             Case 1: $A$ is invertible. Start with the relationship $\adj(A) = \detm(A) A^{-1}$[cite: 111].
        \end{hint}
         \begin{hint}
             If $\lambda$ is an eigenvalue of $A$, what are the eigenvalues of $A^{-1}$? [cite: 112]
         \end{hint}
         \begin{hint}
              If $M = c N$, how do the eigenvalues of $M$ relate to the eigenvalues of $N$? Use this with $M = \adj(A)$, $N=A^{-1}$, and $c = \detm(A)$ to find the eigenvalues of $\adj(A)$ in terms of $\lambda$ and $\detm(A)$[cite: 113].
         \end{hint}
         \begin{hint}
              Case 2: $A$ is singular ($\detm(A)=0$). Assume $\rank(A)=n-1$ (the most interesting singular case)[cite: 114]. What is the rank of $\adj(A)$? [cite: 81, 114]
         \end{hint}
         \begin{hint}
              A matrix of rank 1 has at most one non-zero eigenvalue[cite: 115]. What are most of the eigenvalues of $\adj(A)$ in this case? [cite: 116]
         \end{hint}
         \begin{hint}
              Let $A\mathbf{v} = \lambda \mathbf{v}$ with $\lambda \neq 0$. Consider $A \adj(A) = 0$. Multiply by $\mathbf{v}$ on the right. Alternatively use $\adj(A) A = 0$. Multiply by $\mathbf{v}$ on the right: $\adj(A) A \mathbf{v} = \mathbf{0}$. Substitute $A\mathbf{v}=\lambda \mathbf{v}$[cite: 117, 118].
         \end{hint}
          \begin{hint}
              You should get $\lambda \adj(A) \mathbf{v} = \mathbf{0}$[cite: 118]. Since $\lambda \neq 0$, what does this say about $\adj(A)\mathbf{v}$? What eigenvalue of $\adj(A)$ does the eigenvector $\mathbf{v}$ (corresponding to $\lambda \neq 0$ of $A$) correspond to? [cite: 119, 120]
          \end{hint}
          \begin{hint}
              What about the eigenvalue $\lambda=0$ of $A$? Let $A\mathbf{w} = \mathbf{0}$ where $\mathbf{w}$ is in the nullspace of $A$. Since $\rank(\adj(A))=1$ and its columns are in the nullspace of $A$, the image of $\adj(A)$ is spanned by $\mathbf{w}$[cite: 124, 125]. Thus $\adj(A)\mathbf{w}$ must be a multiple of $\mathbf{w}$. $\adj(A)\mathbf{w} = \mu \mathbf{w}$. This $\mu$ is the potentially non-zero eigenvalue of $\adj(A)$[cite: 126]. Can we determine $\mu$? It's given as $\tr(\adj(A))$[cite: 126].
          \end{hint}
         \begin{hint}
             What if $\rank(A) \le n-2$? What is $\adj(A)$? What are its eigenvalues? [cite: 127]
         \end{hint}
     
\end{problem}

\section{Challenge / Contest-Style Problems}

\begin{problem}{From Properties to Solutions}
    \begin{question}
        If $A$ is a real $n \times n$ matrix such that $\adj(A) = A^T$, what are the possible values for $\detm(A)$?
    \end{question}
    
        \begin{hint}
            Start by taking the determinant of both sides of the equation $\adj(A) = A^T$.
        \end{hint}
        \begin{hint}
            Use the identities $\detm(\adj(A)) = (\detm A)^{n-1}$ [cite: 57] and $\detm(A^T) = \detm(A)$.
        \end{hint}
        \begin{hint}
            You should arrive at an equation involving only $\detm(A)$ and $n$. Let $d = \detm(A)$. The equation is $d^{n-1} = d$.
        \end{hint}
        \begin{hint}
            Solve the equation $d^{n-1} - d = 0$ or $d(d^{n-2}-1) = 0$ for $d$. What are the possible real solutions for $d$? Consider the cases $n=1$, $n=2$, and $n > 2$.
        \end{hint}
         \begin{hint}
             If $n=1$, $A=(a)$, $\adj(A)=(1)$, $A^T=(a)$. So $a=1$. $\detm(A)=1$. Does $d(d^{1-2}-1)=0$ make sense? Let's use $d^{n-1}=d$. If $n=1$, $d^0=d$, so $1=d$.
         \end{hint}
         \begin{hint}
             If $n=2$, $d(d^0-1)=0$, so $d(1-1)=0$, which is $0=0$. This doesn't restrict $d$. Let's re-evaluate. For $n=2$, $d^2=d$. This identity gives no information for $n=2$.
             Let's try $A\adj(A) = \detm(A)I$. Substitute $\adj(A)=A^T$. So $A A^T = \detm(A)I$. Take determinants: $\detm(A)\detm(A^T) = (\detm A)^n$. $(\detm A)^2 = (\detm A)^n$. $d^2 = d^n$. $d^2(1-d^{n-2})=0$.
             This implies $d=0$ or $d^{n-2}=1$.
             If $n=2$, $d^0=1$, which is $1=1$. Still no restriction. If $A=\begin{pmatrix} a&b\\c&d \end{pmatrix}$, $A^T=\begin{pmatrix} a&c\\b&d \end{pmatrix}$, $\adj(A)=\begin{pmatrix} d&-b\\-c&a \end{pmatrix}$. So we need $d=a, -b=c, -c=b, a=d$. This means $a=d$ and $c=-b$. Matrix is $A=\begin{pmatrix} a&b\\-b&a \end{pmatrix}$. $\detm(A)=a^2+b^2$. Can this be any non-negative real number? Yes. So for $n=2$, $\detm(A)$ can be any value $\ge 0$.
         \end{hint}
         \begin{hint}
             If $n>2$, $d=0$ or $d^{n-2}=1$. Since $A$ is real, $d^{n-2}=1$ implies $d=1$ or (if $n$ is even) $d=-1$.
         \end{hint}
         \begin{hint}
              Summarize the possible values for $\detm(A)$ based on $n$.
         \end{hint}
    

    \begin{question}
        Prove the result cited in the document: If $A^2=A$ (and $A$ is $n \times n$) and $\rank(A)=n-1$, then $\adj(A) = I-A$[cite: 262].
    \end{question}
    
        \begin{hint}
            Since $A^2=A$ and $A \neq I$ (because $\rank(A)=n-1 < n$), $A$ must be singular. Why? [cite: 240] So $\detm(A)=0$.
        \end{hint}
        \begin{hint}
            What is the rank of $\adj(A)$ if $\rank(A)=n-1$? [cite: 81, 243]
        \end{hint}
        \begin{hint}
            What is the rank of $I-A$? Recall that $A$ and $I-A$ are complementary projections. Use the property $\rank(P) + \rank(I-P) = n$ for a projection $P$. [cite: 244]
        \end{hint}
        \begin{hint}
            So, both $\adj(A)$ and $I-A$ have rank 1. We need to show they are equal, not just proportional.
        \end{hint}
        \begin{hint}
            Consider the product $A \adj(A)$. What is it equal to, since $\detm(A)=0$? [cite: 39]
        \end{hint}
        \begin{hint}
            Now consider the product $A(I-A)$. Expand it using $A^2=A$. Is it equal to $A \adj(A)$? [cite: 246] This shows columns of $\adj(A)$ and $I-A$ are in the nullspace of $A$.
        \end{hint}
        \begin{hint}
            Since $\rank(A)=n-1$, the nullspace of $A$ is 1-dimensional[cite: 85]. Both $\adj(A)$ and $I-A$ are rank 1 matrices whose columns span this nullspace. This means $\adj(A) = k(I-A)$ for some scalar $k$[cite: 249].
        \end{hint}
        \begin{hint}
            How can we find $k$? Take the trace of both sides: $\tr(\adj(A)) = k \tr(I-A)$[cite: 250].
        \end{hint}
        \begin{hint}
            What are the eigenvalues of an idempotent matrix $A$ with rank $n-1$?[cite: 184, 251]. How many eigenvalues are 1 and how many are 0?
        \end{hint}
        \begin{hint}
            Use the eigenvalues to calculate $\tr(A)$. Then calculate $\tr(I-A) = \tr(I) - \tr(A)$[cite: 251, 252].
        \end{hint}
        \begin{hint}
            Recall from Problem 8.1.3 (or re-derive using characteristic polynomial coefficients) that the non-zero eigenvalue of $\adj(A)$ (when $\rank(A)=n-1$) is $\tr(\adj(A))$[cite: 126]. Also, the coefficient $c_1$ of $\lambda$ in the characteristic polynomial $p(\lambda)=\detm(\lambda I - A)$ is related to $\tr(\adj(A))$[cite: 257].
        \end{hint}
        \begin{hint}
            The characteristic polynomial for $A$ is $p(\lambda) = \lambda^1 (\lambda-1)^{n-1}$[cite: 255]. Find the coefficient of $\lambda^1$ in this polynomial by expanding $(\lambda-1)^{n-1}$ using the binomial theorem or by differentiation[cite: 259].
        \end{hint}
        \begin{hint}
            Relate this coefficient $c_1$ to $\tr(\adj(A))$ using $c_1 = (-1)^{n-1} \tr(\adj(A))$[cite: 257, 260]. Show that $\tr(\adj(A)) = 1$.
        \end{hint}
        \begin{hint}
            Substitute $\tr(\adj(A))=1$ and $\tr(I-A)=1$ into the equation from Hint 8: $\tr(\adj(A)) = k \tr(I-A)$. What must $k$ be? [cite: 261]
        \end{hint}
        \begin{hint}
            Since $k=1$, we have $\adj(A) = I-A$[cite: 262].
        \end{hint}
    

    % Add more challenge problems as needed, e.g., on self-adjugate, block adjugate (maybe a specific entry).
    \begin{question} Find all $2 \times 2$ real matrices $A$ such that $\adj(A)=A$[cite: 207].
    \end{question}
     
        \begin{hint}
             Let $A = \begin{pmatrix} a & b \\ c & d \end{pmatrix}$. Then $\adj(A) = \begin{pmatrix} d & -b \\ -c & a \end{pmatrix}$[cite: 15]. Set $A=\adj(A)$. What equations must $a,b,c,d$ satisfy?
        \end{hint}
         \begin{hint}
              The equations are $a=d$, $b=-b$, $c=-c$, $d=a$. What do $b=-b$ and $c=-c$ imply about $b$ and $c$?
         \end{hint}
         \begin{hint}
              So, $b=0$ and $c=0$. The matrix must be diagonal, $A=\begin{pmatrix} a & 0 \\ 0 & d \end{pmatrix}$. The condition $a=d$ still holds. So $A=\begin{pmatrix} a & 0 \\ 0 & a \end{pmatrix} = aI_2$.
         \end{hint}
         \begin{hint}
              Now, let's check the general analysis. If $\adj(A)=A$, then $\detm(A)$ must satisfy $d(d^{n-2}-1)=0$. For $n=2$, this is $d(d^0-1)=0 \implies d(1-1)=0 \implies 0=0$. This gave no restriction on $\detm(A)$.
              However, we also have $A\adj(A)=\detm(A)I$. If $\adj(A)=A$, then $A^2 = \detm(A)I$.
         \end{hint}
         \begin{hint}
             Apply $A^2 = \detm(A)I$ to our candidate solution $A=aI_2$.
             $(aI_2)^2 = a^2 I_2$.
             $\detm(aI_2) I_2 = a^2 I_2$.
             So $a^2 I_2 = a^2 I_2$. This holds for any scalar $a$.
         \end{hint}
         \begin{hint}
              Therefore, any matrix of the form $A=aI_2 = \begin{pmatrix} a & 0 \\ 0 & a \end{pmatrix}$ is a solution for $n=2$.
         \end{hint}
    


\end{problem}

\section{Edge Cases}

\begin{problem}{Handling Extremes}
    \begin{question}
        Let $A = (a)$ be a $1 \times 1$ matrix. Verify explicitly that $\adj(A)=(1)$ and that $A \adj(A) = \detm(A) I_1$[cite: 300, 301, 302]. Why is $\adj((0))=(1)$ not zero?
    \end{question}
    
        \begin{hint}
            For $A=(a)$, the cofactor matrix $C$ has entry $C_{11} = (-1)^{1+1} M_{11}$[cite: 300].
        \end{hint}
        \begin{hint}
            The minor $M_{11}$ is the determinant of the matrix obtained by removing row 1 and column 1. This is the $0 \times 0$ matrix. By convention, what is its determinant? [cite: 300]
        \end{hint}
        \begin{hint}
            Calculate $C_{11}$ using the determinant from Hint 2. This gives the cofactor matrix $C=(C_{11})$.
        \end{hint}
        \begin{hint}
            The adjugate is the transpose of the cofactor matrix: $\adj(A) = C^T$[cite: 8, 301]. What is the transpose of a $1 \times 1$ matrix?
        \end{hint}
        \begin{hint}
            Now compute the product $A \adj(A) = (a)(1)$[cite: 302].
        \end{hint}
         \begin{hint}
             Compute the right side $\detm(A) I_1$. What is $\detm((a))$? What is $I_1$? [cite: 302] Compare both sides.
         \end{hint}
         \begin{hint}
             The fact that $\adj((0))=(1)$ follows directly from the definition involving the $0 \times 0$ determinant being 1[cite: 301, 304]. It seems counterintuitive but is required for consistency, especially for the identity $A \adj(A) = \detm(A) I$.
         \end{hint}
    

    \begin{question}
        Verify that $\adj(I_n) = I_n$ for any $n \ge 1$[cite: 141, 307].
    \end{question}
    
        \begin{hint}
            We need the $(i, j)$-th entry of $\adj(I_n)$, which is $C_{ji}(I_n) = (-1)^{j+i} M_{ji}(I_n)$[cite: 9].
        \end{hint}
         \begin{hint}
             Consider the minor $M_{ji}(I_n)$. This is the determinant of $I_n$ with row $j$ and column $i$ removed.
         \end{hint}
         \begin{hint}
             Case 1: $j=i$. The minor $M_{ii}(I_n)$ is the determinant of $I_n$ with row $i$ and column $i$ removed. What matrix is this? What is its determinant? [cite: 305]
         \end{hint}
         \begin{hint}
             Case 2: $j \neq i$. The minor $M_{ji}(I_n)$ is the determinant of $I_n$ with row $j$ and column $i$ removed. Can you show this resulting matrix must have a row or column of all zeros? (Consider the $j$-th row if $j>i$, or the $i$-th column if $i>j$). What is the determinant of a matrix with a zero row or column? [cite: 305]
         \end{hint}
         \begin{hint}
             So, $M_{ji}(I_n) = 1$ if $j=i$ and $0$ if $j \neq i$.
         \end{hint}
         \begin{hint}
             Now find the cofactor $C_{ji}(I_n) = (-1)^{j+i} M_{ji}(I_n)$. Does the sign matter when $M_{ji}=0$? What is $C_{ii}(I_n)$?
         \end{hint}
         \begin{hint}
             This shows that the entry $(i, j)$ of $\adj(I_n)$, which is $C_{ji}(I_n)$, is 1 if $i=j$ and 0 if $i \neq j$. What matrix has these entries? [cite: 307]
         \end{hint}
    
\end{problem}

\end{document}